\documentclass[12pt]{article}
\usepackage[utf8]{inputenc}
\usepackage{amsfonts}
\usepackage{graphicx}
\usepackage{amsmath}
\usepackage{amssymb}
\usepackage{multirow}
\usepackage{array}
\usepackage{float}
\usepackage{commath}
\usepackage[spanish]{babel}
\usepackage{ragged2e}
\usepackage{natbib}
\usepackage{glossaries}
\usepackage{listings}
\usepackage{color}
\usepackage{verbatim}
\usepackage{tikz}
\usepackage{pgfgantt}
\usepackage{fancyhdr}
\usepackage{url}
\def\UrlBreaks{\do\/\do-}
\usepackage{booktabs}
\usepackage{longtable}
\usepackage{siunitx}
\usepackage{tabularx}


\pagestyle{fancy}
\fancyhf{}
\lhead{\rightmark}
\rfoot{Página \thepage}

\begin{document}

\begin{titlepage}

\begin{center}
\begin{figure}
\begin{minipage}{.22\textwidth}
  \centering
  \includegraphics[scale=0.05]{ipn.png}
\end{minipage}%
\begin{minipage}{.56\textwidth}
\begin{center}
\Large{Instituto Politécnico Nacional}\par
\vspace{7mm}
Escuela Superior de Cómputo
\end{center}
\end{minipage}%
\begin{minipage}{.22\textwidth}
  \centering
  \includegraphics[scale=0.5]{escom.jpg}
\end{minipage}
\end{figure}
\newcommand{\addfigure}[4]{
        \begin{figure}[htbp!]
            \begin{center}	
                \fbox{\includegraphics[width=#1\textwidth]{#2}}
                \caption{#4}
                \label{#3}
            \end{center}
        \end{figure}
  }
\large{\hspace{1mm}}\par
\vspace{5mm}
\large{MEDIA SOFT}\par
\vspace{30mm}
\large{\textbf{\textit{SISTEMA DE GESTIÓN PARA LA RENTA DE ESPACIOS PUBLICITARIOS.}}}\par
\vspace{15mm}
\vspace {15mm}
\large{\textbf{Presentan:}}\par
\vspace {5mm}
\large{Bravo Torres Alejandro}\par
\large{Rodríguez García Daniela}\par
\large{Rojas Calero Kevin}\par
\large{Santillán Yáñez Joel}\par
\vspace{10mm}
\end{center} 
\end{titlepage}
\newpage
\pagenumbering{roman}
\tableofcontents
\newpage
\pagenumbering{arabic}
\section{Planteamiento del proyecto}
\subsection{Resumen}
Espectaculares S.A. vino a nosotros comentándonos acerca de los ligeros inconvenientes que ya presenta el sistema de administración de recursos empresariales (ERP) que actualmente posee: es insuficiente para la cantidad de procesos que ellos manejan, y varios de éstos ni siquieran son capaces de administrarse vía ERP.

El presente proyecto, se desarrolla en busca de ofrecer un sistema para la correcta gestión de la renta de los espacios publicitarios que ofertan.

\section{Especificación de requerimientos}

\subsection{Requerimientos funcionales del sistema}

{\small
\begin{longtable}[H]{m{2cm}m{4cm}m{6cm}}
    \caption{\normalsize{Requerimientos funcionales del sistema}}\\
    
    \toprule
    
    \centering \textbf{ID} & \centering  \textbf{Nombre} & \centering \small \textbf{Descripción} \tabularnewline
    \midrule
    \textit{RF1} & \textbf{Visualización de las coordenadas de los espacios publicitarios} & La aplicación deberá permitir visualizar al cliente la ubicación por coordenadas de cada uno de los espacios publicitarios que posee (mediante un mapa). \tabularnewline
    \textit{RF2} & \textbf{Visualización de la dirección del espacio publicitario} & La aplicación deberá permitir visualizar la dirección de cada uno de los espacios publicitarios que tiene a su disposición que deberá tener los siguientes campos: código postal, calle, número exterior, colonia, delegación, estado* y referencias. \tabularnewline
    \textit{RF3} & \textbf{Visualización del estado de renta del espacio publicitario} & La aplicación deberá permitir visualizar  el estado de cada uno de los espacios publicitarios que se ofrecen, que deberán ser los siguientes: ocupado, desocupado. \tabularnewline
    \textit{RF4} & \textbf{Visualización  del estado de mantenimiento del espacio publicitario} & La aplicación deberá permitir la correcta visualización del estado de mantenimiento del espacio publicitario, que se limitan a: en matenimiento, requiere mantenimiento preventino, requiere mantenimiento correctivo, en buenas condiciones. \tabularnewline
    \textit{RF5} & \textbf{Visualización del inicio del período de renta} & Deberá ser posible observar la fecha de inicio del período de renta de un espectacular. La fecha tendrá el siguiente formato: día/mes/año. \tabularnewline
    \textit{RF6} & \textbf{Visualización de término del período de renta} & Deberá ser posible observar la fecha de término del período de renta de un espectacular. La fecha tendrá el siguiente formato: día/mes/año.
    \tabularnewline
    \textit{RF7} & \textbf{Visualización del responsable del contrato de renta} & La aplicación deberá otorgar el nombre de la persona encargada del contrato de arrendamiento. \tabularnewline
    \textit{RF8} & \textbf{Visualización de la categoría a la que pertenece el espacio publicitario} & Deberá ser posible mirar la categoría a la que pertenece el espectacular, pudiendo tratarse de: V.I.P., vía primaria y vía secundaria. \tabularnewline
    \textit{RF9} & \textbf{Visualización del impacto del espacio publicitario} & La aplicación deberá otorgar números esperados de personas que miran (en promedio y con base a un estudio de mercado previamente realizado y cargado a ella) el espectacular. \tabularnewline
    \textit{RF10} & \textbf{Alta de un nuevo espacio publicitario} & La aplicación deberá permitir el ingreso de un nuevo espectacular rellenando los campos correspondientes: ubicación en coordenadas, dirección de la ubicación (que abarca código postal, calle, número exterior, colonia, delegación, estado* y referencias), fecha de inicio de contrato de arrendamiento, fecha de término de contrato de arrendamiento, responsable del contrato de arrendamiento, categoría del espectacular, impacto promedio, medio y formato. \tabularnewline
    \textit{RF11} & \textbf{Cambio a un espacio publicitario} & La aplicación deberá otorgar la opción para hacer uno o varios cambios a cualesquiera de los campos asociados (mencionados anteriormente) a un espectacular previamente registrado, sin que se afecte la información que no sea especificada a realizarse un cambio (ni del propio espectacular u otro). \tabularnewline
    \textit{RF12} & \textbf{Baja de un espacio publicitario} & La aplicación deberá otorgar la opción de eliminar los campos asociados a un espectacular previamente registrado. \tabularnewline
    \textit{RF13} & \textbf{Notificación de mantenimiento} & La aplicación deberá enviar un mensaje de correo electrónico, a la dirección asociada al encargado de mantemiento asignado al espectacular que requiere mantenimiento tres días antes de éste se encuentre contemplado a realizarse. \tabularnewline
    \textit{RF14} & \textbf{Notificación de desmontaje} & La aplicación deberá enviar un mensaje de correo electrónico, a la dirección asociada al encargado de mantemiento asignado al espectacular que va a desmontarse tres días antes de éste se encuentre contemplado a realizarse. \tabularnewline
    \textit{RF15} & \textbf{Notificación de seguro por vencer} & La aplicación deberá enviar un mensaje de correo electrónico, a la dirección asociada al agente de seguros asignado al espectacular del cual se vencerá el seguro, tres semanas antes de que esto suceda. \tabularnewline
    \bottomrule
    
\end{longtable}
}
\newpage
\subsection{Requerimientos no funcionales del sistema}

{\small
\begin{longtable}[H]{m{2cm}m{4cm}m{6cm}}
    \caption{\normalsize{Requerimientos no funcionales del sistema}}\\
    
    \toprule
    
    \centering \textbf{ID} & \centering  \textbf{Nombre} & \centering \small \textbf{Descripción} \tabularnewline
    \midrule
    \textit{RNF1} & \textbf{Responsiva} & La aplicación deberá ser diseñada para su correcta visualización tanto en navegadores para dispositivos móviles como de escritorio. \tabularnewline
    \textit{RNF2} & \textbf{Condiciones de acceso} & La aplicación deberá permitir el acceso al usuario, siempre y cuando no se presenten un fallo. \tabularnewline
    \textit{RNF3} & \textbf{Duración de fallos} & El promedio de tiempo de duración en las fallas que se presenten en la aplicación no podrá ser mayor a 2 horas. \tabularnewline
    \textit{RNF4} & \textbf{Capacidad de acceso} & La aplicación debe ser capaz de operar adecuadamente con 1,000 usuarios (entre empleados y clientes)) utilizándola a la vez. \tabularnewline
    \textit{RNF5} & \textbf{Compatibilidad del navegador} & La aplicación deberá ser visualizada correctamente por los siguientes navegadores: Google Chrome, Mozilla Firefoz, Opera y Safari. \tabularnewline
    \bottomrule
\end{longtable}
}

\end{document}
