%=========================================================
\chapter{Glosario}	
Se procede a enunciar términos recurrentemente utilizados a lo largo del documento en tabla~\ref{tbl:glosario}
\begin{longtable}[H]{m{4cm}m{8cm}}
\toprule
\centering \textbf{Término} & \centering  \textbf{Descripción} \tabularnewline
\midrule

\textbf{Agente de capital humano} & Empleado de la empresa, perteneciente al departamento Capital Humano, colaborador del jefe de capital humano. \tabularnewline

\textbf{Agente de jurídico} & Empleado de la empresa, perteneciente al departamento Jurídico, colaborador del titular de jurídico. \tabularnewline

\textbf{Agente de ventas} & Empleado de la empresa, perteneciente al departamento de Ventas, colaborador del gerente de ventas. \tabularnewline

\textbf{Cartera de clientes} & Listado de clientes de la empresa. \tabularnewline
<<<<<<< HEAD
\bottomrule
=======

\textbf{Cliente} & Persona física o moral que ha solicitado por lo menos un presupuesto de contrato de renta dentro del plazo de tolerancia vigente o rentado algún(os) espectacular(es) en algún momento desde la consolidación de la empresa hasta el día en que se implemente el sistema. \tabularnewline
>>>>>>> master

\textbf{Contrato de renta/contrato de arrendamiento} & Contrato generado por el agente de ventas ó gerente de ventas que atendió al cliente y cerró el trato con él, por un período de renta específico y por concepto de un tipo(s) de espectacular(es) específico(s). \tabularnewline

\textbf{Colaborador} & Empleado que trabaja con otros en la realización de una tarea común de la empresa. \tabularnewline

<<<<<<< HEAD
\textbf{Cliente} & Persona física o moral que ha solicitado por lo menos un presupuesto de contrato de renta dentro del plazo de tolerancia vigente o rentado algún(os) espectacular(es) en algún momento desde la consolidación de la empresa hasta el día en que se implemente el sistema. \tabularnewline

\textbf{Credenciales} & Identificador asociado a un empleado y su contraseña. \tabularnewline

=======
\textbf{Credenciales} & Identificador asociado a un empleado y su contraseña. \tabularnewline

\textbf{Cuadrilla de mantenimiento/cuadrilla} & Grupo de agentes de infraestructura que trabajan en una determinada zona y horario que se reportan con su líder de cuadrilla. \tabularnewline


\textbf{Eliminación lógica} & Acción que realizan los jefes de departamento y que tiene que ver con la colocación de un estado especial que coloca como inaccesible por agentes del departamento, ya sea a un empleado, un cliente, un espectacular, etc. \tabularnewline

>>>>>>> master
\textbf{Empleado} &  Comprende al personal cuyas funciones son de jefatura, ventas, jurídico y capital humano contratado por la empresa. Excluye al personal contratado por terceros con los que la empresa labora (publicidad, por ejemplo).\tabularnewline

\textbf{Empresa} & Hace referencia al cliente que solicitó el sistema: ESPECTACULARES S.A. de C.V.\tabularnewline

<<<<<<< HEAD
=======
\textbf{ERP} & Siglas que hacen referencia al sistema de administración de recursos empresariales que posee actualmente la empresa. \tabularnewline

>>>>>>> master
\textbf{Espectacular} & Espacio publicitario que arrienda la empresa. \tabularnewline

\textbf{Gerente de ventas} & Empleado de la empresa, perteneciente al departamento de Ventas, jefe de dicho departamento. \tabularnewline

\textbf{Identificador} & Referente a un registro alfanumérico único asignado a un empleado o espectacular. \tabularnewline

\textbf{Jefe} & Empleado que lidera el trabajo o actividades a realizar de los colaboradores pertenecientes a su departamento. \tabularnewline

\textbf{Jefe de capital humano} & Empleado de la empresa, perteneciente al departamento de Capital Humano, jefe de dicho departamento. \tabularnewline

<<<<<<< HEAD
=======
\textbf{Líder de cuadrilla} & Agente de insfraestructura que lidera una cuadrilla y se reporta directamente con el jefe de infraestructura. \tabularnewline

>>>>>>> master
\textbf{Período de renta} & Referente al tiempo en que un cliente desea establecer un contrato de renta (de al menos un mes) con la empresa. \tabularnewline

\textbf{Permiso(s)} & Referente a alguno de los permisos asociados a un espectacular (pudiendo tratarse de los emitidos por la SSP, SPC , SOBSE o los tres). \tabularnewline

\textbf{Plantilla} & Conjunto de elementos fijos de la empresa (pudiendo tratarse de una plantilla de empleados, espectaculares, etc.). \tabularnewline

\textbf{Plazo de tolerancia} & Plazo concedido por la empresa al cliente, que transcurre desde la solicitud de presupuesto hasta tres meses después de solicitado éste para respetar el precio acordado con el cliente.\tabularnewline

\textbf{Póliza} & Referente a la póliza de seguros que cubre a un espectacular. \tabularnewline

\textbf{Presupuesto de contrato de renta} & Referente a una solicitud manifestada por el cliente que requiera la renta de un(os) espectacular(es) para un nuevo proyecto a la empresa con base al grado de impacto, tipo, localización estimada, vistas, medio, medidas y precio máximo a pagar por ello.\tabularnewline

\textbf{Publicidad} & Terceros con los que la empresa labora, que no pertenecen a ésta y se dedican a realizar los estudios de mercado que requieren algunos de los clientes de la empresa.\tabularnewline

\textbf{Solicitud de renovación del contrato de renta} & Referente a una solicitud manifestada por el cliente que requiera una ampliación del período de renta del (los) espectacular(es) que cumplan con las especificaciones requeridas por éste (grado de impacto, medidas, medio, vistas, ubicación, tipo y precio) comparados con los que actualmente renta.\tabularnewline

\textbf{SOBSE} & Siglas que hacen referencia a la Secretaría de Obras y Servicios de la Ciudad de México. \tabularnewline

\textbf{SPC} & Siglas que hacen referencia a la Secretaría de Protección Civil de la Ciudad de México. \tabularnewline

\textbf{SSP} & Siglas que hacen referencia a la Secretaría de Seguridad Pública de la Ciudad de México. \tabularnewline

\textbf{Titular jurídico} & Empleado de la empresa, perteneciente al departamento Jurídico, jefe de dicho departamento. \tabularnewline

<<<<<<< HEAD
\textbf{Cartera de clientes} & Listado de clientes de la empresa (con sus respectivos datos de contacto). \tabularnewline

\textbf{Permisos} & Referente a cualesquiera (o a todos) los permisos elaborados por la SSP, SPC y SOBSE asociados a un espectacular. \tabularnewline

\textbf{ERP} & Siglas que hacen referencia al sistema de administración de recursos empresariales que posee actualmente la empresa. \tabularnewline

\textbf{Cuadrilla de mantenimiento/cuadrilla} & Grupo de agentes de infraestructura que trabajan en una determinada zona y horario que se reportan con su líder de cuadrilla. \tabularnewline

\textbf{Líder de cuadrilla} & Agente de insfraestructura que lidera una cuadrilla y se reporta directamente con el jefe de infraestructura. \tabularnewline

\textbf{Eliminación lógica} & Acción que realizan los jefes de departamento y que tiene que ver con la colocación de un estado especial que coloca como inaccesible por agentes del departamento, ya sea a un empleado, un cliente, un espectacular, etc. \tabularnewline

\label{tbl:glosario}
    \caption{Glosario de términos}
    \bottomrule
=======
\textbf{Permisos} & Referente a cualesquiera (o a todos) los permisos elaborados por la SSP, SPC y SOBSE asociados a un espectacular. \tabularnewline
\bottomrule

\caption{Glosario de términos}
\label{tbl:glosario}
>>>>>>> master
\end{longtable}