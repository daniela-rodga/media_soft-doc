\documentclass[12pt]{article}
\usepackage[utf8]{inputenc}
\usepackage{amsfonts}
\usepackage{graphicx}
\usepackage{amsmath}
\usepackage{amssymb}
\usepackage{multirow}
\usepackage{array}
\usepackage{float}
\usepackage{commath}
\usepackage[spanish]{babel}
\usepackage{ragged2e}
\usepackage{natbib}
\usepackage{glossaries}
\usepackage{listings}
\usepackage{color}
\usepackage{verbatim}
\usepackage{tikz}
\usepackage{pgfgantt}
\usepackage{fancyhdr}
\usepackage{url}
\def\UrlBreaks{\do\/\do-}
\usepackage{booktabs}
\usepackage{longtable}
\usepackage{siunitx}
\usepackage{tabularx}


\pagestyle{fancy}
\fancyhf{}
\lhead{\rightmark}
\rfoot{Página \thepage}

\begin{document}

\begin{titlepage}

\begin{center}
\begin{figure}
\begin{minipage}{.22\textwidth}
  \centering
  \includegraphics[scale=0.05]{ipn.png}
\end{minipage}%
\begin{minipage}{.56\textwidth}
\begin{center}
\Large{Instituto Politécnico Nacional}\par
\vspace{7mm}
Escuela Superior de Cómputo
\end{center}
\end{minipage}%
\begin{minipage}{.22\textwidth}
  \centering
  \includegraphics[scale=0.5]{escom.jpg}
\end{minipage}
\end{figure}
\newcommand{\addfigure}[4]{
        \begin{figure}[htbp!]
            \begin{center}	
                \fbox{\includegraphics[width=#1\textwidth]{#2}}
                \caption{#4}
                \label{#3}
            \end{center}
        \end{figure}
  }
\large{\hspace{1mm}}\par
\vspace{5mm}
\large{MEDIA SOFT}\par
\vspace{30mm}
\large{\textbf{\textit{SISTEMA DE GESTIÓN PARA LA RENTA DE ESPACIOS PUBLICITARIOS.}}}\par
\vspace{15mm}
\vspace {15mm}
\large{\textbf{Presentan:}}\par
\vspace {5mm}
\large{Bravo Torres Alejandro}\par
\large{Rodríguez García Daniela}\par
\large{Rojas Calero Kevin}\par
\large{Santillán Yáñez Joel}\par
\vspace{10mm}
\end{center} 
\end{titlepage}
\newpage
\pagenumbering{roman}
\tableofcontents
\newpage
\pagenumbering{arabic}
\section{Planteamiento del proyecto}
\subsection{Resumen}
Espectaculares S.A. vino a nosotros comentándonos acerca de los ligeros inconvenientes que ya presenta el sistema de administración de recursos empresariales (ERP) que actualmente posee: es insuficiente para la cantidad de procesos que ellos manejan, y varios de éstos ni siquieran son capaces de administrarse vía ERP.

El presente proyecto, se desarrolla en busca de ofrecer un sistema para la correcta gestión de la renta de los espacios publicitarios que ofertan.
\section{Interacción del usuario}
\subsection{Procesos del negocio}
\begin{longtable}[H]{m{4cm}m{8cm}}
    \toprule
    
    \centering \textbf{Usuario} & \centering  \textbf{Procesos} \tabularnewline
    \midrule
    \textbf{Gerente de Ventas} & Alta de un nuevo cliente, cambios a un cliente existente, búsqueda de un cliente, visualización datos de todo el listado de clientes, eliminación de datos de un cliente perdido.\tabularnewline
    \textbf{Agente de ventas} & Alta de un nuevo cliente, cambios a un cliente existente, búsqueda de un cliente, visualización datos de todo el listado de clientes.\tabularnewline
    \textbf{Titular de Jurídico} & Alta de un nuevo permiso (cualesquiera de los tres emitidos por: Secretaría de Seguridad Pública de la Ciudad de México, Secretaría de Obras y Servicios de la Ciudad de México, Secretaría de Protección Civil de la Ciudad de México), cambios a un permiso registrado (cualesquiera de los tres emitidos por: Secretaría de Seguridad Pública de la Ciudad de México, Secretaría de Obras y Servicios de la Ciudad de México, Secretaría de Protección Civil de la Ciudad de México), búsqueda de un permiso registradi (cualesquiera de los tres emitidos por: Secretaría de Seguridad Pública de la Ciudad de México, Secretaría de Obras y Servicios de la Ciudad de México, Secretaría de Protección Civil de la Ciudad de México), visualización datos de todo el listado de permisos vigentes asociado a un espectacular, alta de una nueva póliza de seguros de un espectacular, cambios  a una póliza de seguros de un espectacular,  visualización datos de póliza de seguros vigente asociada a un espectacular, eliminación de datos de una póliza de seguros, eliminación de datos de un permiso cualesquiera de los tres emitidos por: Secretaría de Seguridad Pública de la Ciudad de México, Secretaría de Obras y Servicios de la Ciudad de México, Secretaría de Protección Civil de la Ciudad de México).\tabularnewline
    \textbf{Agente jurídico} & Alta de un nuevo permiso (cualesquiera de los tres emitidos por: Secretaría de Seguridad Pública de la Ciudad de México, Secretaría de Obras y Servicios de la Ciudad de México, Secretaría de Protección Civil de la Ciudad de México), cambios a un permiso registrado (cualesquiera de los tres emitidos por: Secretaría de Seguridad Pública de la Ciudad de México, Secretaría de Obras y Servicios de la Ciudad de México, Secretaría de Protección Civil de la Ciudad de México), búsqueda de un permiso registradi (cualesquiera de los tres emitidos por: Secretaría de Seguridad Pública de la Ciudad de México, Secretaría de Obras y Servicios de la Ciudad de México, Secretaría de Protección Civil de la Ciudad de México), visualización datos de todo el listado de permisos vigentes asociado a un espectacular, alta de una nueva póliza de seguros de un espectacular, cambios  a una póliza de seguros de un espectacular,  visualización datos de póliza de seguros vigente asociada a un espectacular.\tabularnewline
    \textbf{Jefe de Infraestructura} & Alta de un nuevo espectacular, cambios a un espectacular existente, búsqueda de un espectacular, visualización datos de todo el listado de espectaculares, eliminación de un espectacular.\tabularnewline
    \textbf{Agente de Infraestructura} & Alta de un nuevo espectacular, cambios a un espectacular existente, búsqueda de un espectacular, visualización datos de todo el listado de espectaculares.\tabularnewline
    \textbf{Jefe de Capital Humano} & Alta de un nuevo empleado, cambios a un empleado existente, visualización datos de todo el listado de empleados, eliminación de un empleado.\tabularnewline
    \textbf{Agente de Capital Humano} & Alta de un nuevo empleado, cambios a un empleado existente, visualización datos de todo el listado de empleados.
    \caption{Usuarios y procesos contemplados}
\label{tbl:listaUP}
   \bottomrule
\end{longtable}

\section{Especificación de requerimientos}

\subsection{Requerimientos funcionales del sistema}

{\small
\begin{longtable}[H]{m{2cm}m{4cm}m{6cm}}
    \caption{\normalsize{Requerimientos funcionales del sistema ACT- 18-09-18 02:38}}\\
    
    \toprule
    
    \centering \textbf{ID} & \centering  \textbf{Nombre} & \centering \small \textbf{Descripción} \tabularnewline
    \midrule
    \textit{RF1} & \textbf{Alta a un nuevo empleado} & La aplicación deberá permitir el ingreso de un nuevo empleado rellenando los campos correspondientes: nombre completo (compuesto por nombre(s), apellido paterno y apellido materno), dirección (que abarca código postal, calle, número exterior, número interior, colonia, delegación, estado* y referencias), teléfono fijo (opcional), teléfono celular (con lada incluida), dirección de correo electrónico, situación (activo, inactivo), departamento (ventas, jurídico, infraestructura, capital humano) y tipo de empleado (agente, jefe).\tabularnewline
     \textit{RF2} & \textbf{Identificador de empleado} & A cada empleado registrado se le asignará un identificador único, que será utilizado para identificarle en todos los procesos subsecuentes que se realicen sobre él.\tabularnewline
     \textit{RF3} & \textbf{Visualización de los datos del empleado} & La aplicación deberá permitir visualizar los datos personales de un empleado específico con base en su identificador.\tabularnewline
     \textit{RF4 (checa RF9)} & \textbf{Cambio a un cliente} & La aplicación deberá otorgar la opción para hacer uno o varios cambios a cualesquiera de los campos asociados a un cliente previamente registrado con base a su identificador, sin que se afecte la información que no sea especificada a realizarse un cambio (ni del propio cliente u otro).\tabularnewline
    \textit{RF5} & \textbf{Alta de un nuevo espectacular} & La aplicación deberá permitir el ingreso de un nuevo espectacular rellenando los campos correspondientes: ubicación en coordenadas, dirección de la ubicación (que abarca código postal, calle, número exterior, colonia, delegación, estado* y referencias), estado, fecha de inicio de contrato de arrendamiento (con formato día/mes/año, opcional), fecha de término de contrato de arrendamiento (con formato día/mes/año, opcional), agente de ventas asignado al contrato de arrendamiento (nombre e identificador, opcional), categoría del espectacular, impacto promedio, medio, vistas, iluminación y medidas.\tabularnewline
    \textit{RF6} & \textbf{Identificador del espectacular} & A cada espectacular registrado se le asignará un identificador único, que será utilizado para identificarle en todos los procesos subsecuentes que se realicen sobre él. \tabularnewline
    \textit{RF7} & \textbf{Visualización de las coordenadas del espectacular} & La aplicación deberá permitir visualizar la ubicación por coordenadas de cada uno de los espacios publicitarios que posee (mediante un mapa). \tabularnewline
    \textit{RF8} & \textbf{Visualización de la dirección del espectacular} & La aplicación deberá permitir visualizar la dirección de cada uno de los espacios publicitarios que tiene a su disposición que deberá tener los siguientes campos: código postal, calle, número exterior, colonia, delegación, estado* y referencias. \tabularnewline
    \textit{RF9(antes RF4 ESTE NO VA EN CASO DE USO!!)} & \textbf{Visualización  del estado del espectacular} & La aplicación deberá permitir la correcta visualización del estado del espectacular, que se limitan a: en matenimiento preventivo, en mantenimiento correctivo, requiere mantenimiento preventivo, requiere mantenimiento correctivo, en instalación, disponible, apartado, en desmontaje, en uso, dañado, descontinuado. \tabularnewline
    \textit{RF10} & \textbf{Visualización del inicio del período de renta} & Deberá ser posible observar la fecha de inicio del período de renta de un espectacular. La fecha tendrá el siguiente formato: día/mes/año. \tabularnewline
    \textit{RF11} & \textbf{Visualización de término del período de renta} & Deberá ser posible observar la fecha de término del período de renta de un espectacular. La fecha tendrá el siguiente formato: día/mes/año.
    \tabularnewline
    \textit{RF12} & \textbf{Visualización del agente de venta} & La aplicación deberá otorgar el nombre del agente de venta asociado a dicho contrato de arrendamiento. \tabularnewline
    \textit{RF13} & \textbf{Visualización de la categoría a la que pertenece el espectacular} & Deberá ser posible mirar la categoría a la que pertenece el espectacular, pudiendo tratarse de: V.I.P., vía primaria y vía secundaria. \tabularnewline
    \textit{RF14} & \textbf{Visualización del impacto del espacio publicitario} & La aplicación deberá otorgar números esperados de personas que miran (en promedio y con base a un estudio de mercado previamente realizado y cargado a ella) el espectacular. \tabularnewline
    \textit{RF15} & \textbf{Visualización del medio del espectacular} & La aplicación deberá permitir la visualización del recurso visual del espectacular, pudiendo ser: pantalla o lona. \tabularnewline
    \textit{RF16} & \textbf{Visualización de las vistas del espectacular} & La aplicación deberá mostrar correctamente el número de vistas que posee un determinado espectacular, pudiendo tratarse de: una vista, dos vistas o tres vistas. \tabularnewline
    \textit{RF17} & \textbf{Visualización de la iluminación del espectacular} & La aplicación deberá permitir la visualización del estado de la iluminación del espectacular siendo éstos únicamente: cuenta con illuminación, no cuenta con iluminación. \tabularnewline
    \textit{RF18} & \textbf{Visualización de las medidas del espectacular} & La aplicación deberá permitir la visualización de las medidas espectacular siendo éstas representadas en metros. \tabularnewline
     \textit{RF19} & \textbf{Cambio a un espectacular} & La aplicación deberá otorgar la opción para hacer uno o varios cambios a cualesquiera de los campos asociados a un espectacular previamente registrado con base a su identificador, sin que se afecte la información que no sea especificada a realizarse un cambio (ni del propio espectacular u otro). \tabularnewline
     \textit{RF20} & \textbf{Alta a un nuevo cliente} & La aplicación deberá permitir el ingreso de un nuevo cliente rellenando los campos correspondientes: nombre completo (compuesto por nombre(s), apellido paterno y apellido materno), dirección (que abarca código postal, calle, número exterior, número interior, colonia, delegación, estado* y referencias), teléfono fijo (opcional), teléfono celular (con lada incluida), dirección de correo electrónico, situación (de antaño, nuevo, perdido) y contratos asociados a éste (opcional).\tabularnewline
     \textit{RF21} & \textbf{Identificador de cliente} & A cada cliente registrado se le asignará un identificador único, que será utilizado para identificarle en todos los procesos subsecuentes que se realicen sobre él.\tabularnewline
     \textit{RF22} & \textbf{Visualización de los datos del cliente} & La aplicación deberá permitir visualizar los datos personales de un cliente específico con base en su identificador.\tabularnewline
     \textit{RF23} & \textbf{Cambio a un cliente} & La aplicación deberá otorgar la opción para hacer uno o varios cambios a cualesquiera de los campos asociados a un cliente previamente registrado con base a su identificador, sin que se afecte la información que no sea especificada a realizarse un cambio (ni del propio cliente u otro).\tabularnewline
     \textit{RF24} & \textbf{Alta de una póliza de seguro} & La aplicación deberá permitir el ingreso de una nueva póliza de seguro rellenando los campos correspondientes: nombre de la aseguradora contratada, número de póliza, fecha de inicio de cobertura de la póliza (con formato día/mes/año), fecha de término de cobertura de la póliza (con formato día/mes/año), cobertura de la póliza (monto y siniestros), estado (pudiendo encontrarse vigente, vencida a cambiar, por vencer y vencida registrada) y agente jurídico que tramitó dicha póliza (nombre e identificador).\tabularnewline
     \textit{RF25} & \textbf{Identificador de una póliza de seguro} & A cada póliza de seguro registrada se le asignará un identificador único, que será utilizado para identificarle en todos los procesos subsecuentes que se realicen sobre ella.& \tabularnewline
     \textit{RF26} & \textbf{Asociar una póliza de seguro con un espectacular} & Al ingresar pólizas de seguro, toda póliza de seguro estará asociada a un espectacular. \tabularnewline
     \textit{RF27} & \textbf{Visualización de una póliza de seguro} & La aplicación deberá permitir visualizar los datos de una póliza de seguro específica con base en su identificador.\tabularnewline
     \textit{RF28} & \textbf{Cambio a una póliza de seguro} & La aplicación deberá otorgar la opción para hacer uno o varios cambios a cualesquiera de los campos asociados a una póliza de seguro previamente registrada con base a su identificador, sin que se afecte la información que no sea especificada a realizarse un cambio (ni de la propia póliza u otra).\tabularnewline
     \textit{RF29} & \textbf{Alta de un permiso de la Secretaría de Seguridad Pública} & La aplicación deberá permitir el ingreso de un nuevo permiso de la Secretaría de Seguridad Pública rellenando los campos correspondientes: número de permiso, fecha de inicio de cobertura del permiso (con formato día/mes/año), fecha de término de cobertura del permiso (con formato día/mes/año), estado (pudiendo encontrarse vigente, vencido a cambiar, por vencer y vencida registrado) y agente jurídico que tramitó dicho permiso (nombre e identificador).\tabularnewline
     \textit{RF30} & \textbf{Identificador de un permiso de la Secretaría de Seguridad Pública} & A permiso de la Secretaría de Seguridad Pública registrado se le asignará un identificador único, que será utilizado para identificarle en todos los procesos subsecuentes que se realicen sobre él. \tabularnewline
     \textit{RF31} & \textbf{Asociar un permiso de la Secretaría de Seguridad Pública} & Al ingresar permisos de la Secretaría de Seguridad Pública, todo permiso estará asociado a un espectacular. \tabularnewline
     \textit{RF32} & \textbf{Visualización de un permiso de la Secretaría de Seguridad Pública} & La aplicación deberá permitir deberá permitir visualizar los datos de un permiso de la Secretaría de Seguridad Pública específica con base en su identificador.\tabularnewline
     \textit{RF33} & \textbf{Cambio a un permiso de la Secretaría de Seguridad Pública} & La aplicación deberá otorgar la opción para hacer uno o varios cambios a cualesquiera de los campos asociados a un permiso de la Secretaría de Seguridad Pública previamente registrado con base a su identificador, sin que se afecte la información que no sea especificada a realizarse un cambio (ni del propio permiso u otro).\tabularnewline
     \textit{RF34} & \textbf{Alta de un permiso de la Secretaría de Protección Civil} & La aplicación deberá permitir el ingreso de un nuevo permiso de la Secretaría de Protección Civil rellenando los campos correspondientes: número de permiso, fecha de inicio de cobertura del permiso (con formato día/mes/año), fecha de término de cobertura del permiso (con formato día/mes/año), estado (pudiendo encontrarse vigente, vencido a cambiar, por vencer y vencida registrado) y agente jurídico que tramitó dicho permiso (nombre e identificador).\tabularnewline
     \textit{RF35} & \textbf{Identificador de un permiso de la Secretaría de Protección Civil} & A permiso de la Secretaría de Protección Civil registrado se le asignará un identificador único, que será utilizado para identificarle en todos los procesos subsecuentes que se realicen sobre él. \tabularnewline
     \textit{RF36} & \textbf{Asociar un permiso de la Secretaría de Protección Civil} & Al ingresar permisos de la Secretaría de Protección Civil, todo permiso estará asociado a un espectacular. \tabularnewline
     \textit{RF37} & \textbf{Visualización de un permiso de la Secretaría de Protección Civil} & La aplicación deberá permitir deberá permitir visualizar los datos de un permiso de la Secretaría de Protección Civil específica con base en su identificador.\tabularnewline
     \textit{RF38} & \textbf{Cambio a un permiso de la Secretaría de Protección Civil} & La aplicación deberá otorgar la opción para hacer uno o varios cambios a cualesquiera de los campos asociados a un permiso de la Secretaría de Protección Civil previamente registrado con base a su identificador, sin que se afecte la información que no sea especificada a realizarse un cambio (ni del propio permiso u otro).\tabularnewline
     \textit{RF39} & \textbf{Alta de un permiso de la Secretaría de Obras y Servicios} & La aplicación deberá permitir el ingreso de un nuevo permiso de la Secretaría de Obras y Servicios rellenando los campos correspondientes: número de permiso, fecha de inicio de cobertura del permiso (con formato día/mes/año), fecha de término de cobertura del permiso (con formato día/mes/año), estado (pudiendo encontrarse vigente, vencido a cambiar, por vencer y vencida registrado) y agente jurídico que tramitó dicho permiso (nombre e identificador).\tabularnewline
     \textit{RF40} & \textbf{Identificador de un permiso de la Secretaría de Obras y Servicios} & A permiso de la Secretaría de Obras y Servicios registrado se le asignará un identificador único, que será utilizado para identificarle en todos los procesos subsecuentes que se realicen sobre él. \tabularnewline
     \textit{RF41} & \textbf{Asociar un permiso de la Secretaría de Obras y Servicios} & Al ingresar permisos de la Secretaría de Obras y Servicios, todo permiso estará asociado a un espectacular. \tabularnewline
     \textit{RF42} & \textbf{Visualización de un permiso de la Secretaría de Obras y Servicios} & La aplicación deberá permitir deberá permitir visualizar los datos de un permiso de la Secretaría de Obras y Servicios específica con base en su identificador.\tabularnewline
     \textit{RF43} & \textbf{Cambio a un permiso de la Secretaría de Obras y Servicios} & La aplicación deberá otorgar la opción para hacer uno o varios cambios a cualesquiera de los campos asociados a un permiso de la Secretaría de Obras y Servicios previamente registrado con base a su identificador, sin que se afecte la información que no sea especificada a realizarse un cambio (ni del propio permiso u otro).\tabularnewline
     \textit{RF44} & \textbf{Visualización del listado de espectaculares} & La aplicación deberá permitir la visualización de la lista completa de los espectaculares registrados en ella en el momento que se pida.\tabularnewline
     \textit{RF45} & \textbf{Visualización del listado de clientes} & La aplicación deberá permitir la visualización de la lista completa de los clientes registrados en ella en el momento que se pida.\tabularnewline
     \textit{RF46} & \textbf{Visualización del listado de empleados} & La aplicación deberá permitir la visualización de la lista completa de los empleados registrados en ella en el momento que se pida.\tabularnewline
     \textit{RF47} & \textbf{Visualización del listado de permisos y pólizas asociados a un espectacular} & La aplicación deberá permitir la visualización de la lista completa de los peermisos y pólizas registrados en ella asociados a un espectacular mediante su identificador en el momento que se pida.\tabularnewline
     \textit{RF48} & \textbf{Póliza vencida registrada}& Únicamente el Titular de Jurídico podrá actualizar el estado de una póliza de seguros asociada a un espectacular a "vencida registrada".\tabularnewline
     \textit{RF49} & \textbf{Permiso vencido registrado}& Únicamente el Titular de Jurídico podrá actualizar el(los) estado(s) de un(varios) permiso(s) (cualesquiera de los tres emitidos por: Secretaría de Seguridad Pública de la Ciudad de México, Secretaría de Obras y Servicios de la Ciudad de México, Secretaría de Protección Civil de la Ciudad de México) asociado a un espectacular a "vencido registrado".\tabularnewline
     \textit{RF49} & \textbf{Espectacular descontinuado} & Únicamente el Jefe de Infraestructura podrá actualizar el estado de un espectacular a "descontinuado".\tabularnewline
     \textit{RF50} & \textbf{Empleado inactivo} & Únicamente el Jefe de Capital Humano podrá actualizar el estado de un empleado de cualesquiera de los departamentos a "inactivo".\tabularnewline
     \textit{RF51} & \textbf{Cliente perdido} & Únicamente el Gerente de Ventas podrá actualizar el estado de un cliente a "perdido".\tabularnewline
     
    \bottomrule
    
\end{longtable}
}
\newpage
\subsection{Requerimientos no funcionales del sistema}

{\small
\begin{longtable}[H]{m{2cm}m{4cm}m{6cm}}
    \caption{\normalsize{Requerimientos no funcionales del sistema}}\\
    
    \toprule
    
    \centering \textbf{ID} & \centering  \textbf{Nombre} & \centering \small \textbf{Descripción} \tabularnewline
    \midrule
    \textit{RNF1} & \textbf{Responsiva} & La aplicación deberá ser diseñada para su correcta visualización tanto en navegadores para dispositivos móviles como de escritorio. \tabularnewline
    \textit{RNF2} & \textbf{Condiciones de acceso} & La aplicación deberá permitir el acceso al usuario, siempre y cuando no se presenten un fallo. \tabularnewline
    \textit{RNF3} & \textbf{Duración de fallos} & El promedio de tiempo de duración en las fallas que se presenten en la aplicación no podrá ser mayor a 2 horas. \tabularnewline
    \textit{RNF4} & \textbf{Capacidad de acceso} & La aplicación debe ser capaz de operar adecuadamente con 1,000 usuarios (entre empleados y clientes)) utilizándola a la vez. \tabularnewline
    \textit{RNF5} & \textbf{Compatibilidad del navegador} & La aplicación deberá ser visualizada correctamente por los siguientes navegadores: Google Chrome, Mozilla Firefox y Safari. \tabularnewline
    \bottomrule
\end{longtable}
}

\end{document}
