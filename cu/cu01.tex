% \IUref{IUAdmPS}{Administrar Planta de Selección}
% \IUref{IUModPS}{Modificar Planta de Selección}
% \IUref{IUEliPS}{Eliminar Planta de Selección}

% 


% Copie este bloque por cada caso de uso:
%-------------------------------------- COMIENZA descripción del caso de uso.

%\begin{UseCase}[archivo de imágen]{UCX}{Nombre del Caso de uso}{
%--------------------------------------
\begin{UseCase}{CU01}{Alta a un nuevo empleado}{
	El Gerente de Recursos Humanos y/o un Agente de Recursos Humanos requiere dar de alta en el sistema a un empleado recién contratado.
	}   
	\UCitem{Nombre}{Alta a un nuevo empleado}
    \UCitem{Descripción}{El Gerente y/o Agente(s) de Recursos Humanos serán capaces de ingresar la información personal referente a un nuevo empleado en la plantilla.}
    \UCitem{Versión}{\color{Gray}0.1}
    \UCitem{Autor}{\color{Gray}Daniela Rodríguez García}
    \UCitem{Supervisa}{\color{Gray}Alejandro Bravo Torres}
    \UCitem{Actor}{\hyperlink{Gerente de Recursos Humanos}{Gerente de Recursos Humanos, Agente de Recursos Humanos}}
    \UCitem{Propósito}{Gerentes de los distintos departamentos serán capaces de conocer la informaci{on persona de su plantilla de trabajo.}
    \UCitem{Entradas}{Nombre completo, dirección, teléfono fijo (opcional), teléfono celular, dirección de correo electrónico, situación, departamento, tipo de empleado.}
    \UCitem{Origen}{Teclado}
    \UCitem{Salidas}{Mensaje de Notificación, visualización de datos del empleado.}
    \UCitem{Destino}{Base de datos.}
    \UCitem{Precondiciones}{El Gerente de Recursos Humanos y/o Agente de Recursos Humanos debe estar registrado en el sistema.}
    \UCitem{Postcondiciones}{El nuevo empleado podrá acceder al sistema con su identificador correspondiente. Los datos del nuevo empleado podrán ser visualizados y modificados.}
    \UCitem{Errores}{El Gerente de Recursos Humanos y/o Agente de Recursos Humanos ingresa datos en un formato distinto al específicados u omite uno de los datos obligatorios.}
    \UCitem{Tipo}{Caso de uso primario}
    \UCitem{Observaciones}{bbrbrbr}
\end{UseCase}
%--------------------------------------
\begin{UCtrayectoria}{Principal}
\UCpaso[\UCactor] Introduce su identificador de Usuario y Contraseña en el sistema vía la  \IUref{IU23}{Pantalla de Control de Acceso}\label{CU17Login}.

\UCpaso[\UCactor] Confirma la operación presionando el botón \IUbutton{Entrar}.

\UCpaso Visualiza la \IUref{IU09}{Administración de Recursos Humanos} Administración de Recursos Humanos
\UCpaso[\UCactor] Ingresa al Menu de Empleados.

\UCpaso[\UCactor] Oprime la opción \IUbutton{Alta de nuevos empleados}.

\UCpaso[\UCactor] Ingresa los datos .

\UCpaso Se despliega la \IUref{IU10}{Confirmación de Alta de Nuevo Empleado} Confirmación de Alta de Nuevo Empleado.

\UCpaso El sistema almacena los datos del nuevo empleado.

\UCpaso [\UCactor] Regresa a la \IUref{IU09}{Administración de Recursos Humanos} Administración de Recursos Humanos.


\end{UCtrayectoria}

%--------------------------------------		
\begin{UCtrayectoriaA}{A}{El Gerente de Recursos Humanos y/o Agente de Recursos Humanos aborta el ingreso del nuevo empleado.}
\UCpaso [\UCactor] Presiona \IUbutton{
Cancelar} y se despliega la \IUref{IU11}{Confirmación de cancelación de Alta de Nuevo Empleado}
\UCpaso El sistema regresa a la Pantalla de Administración de Recursos Humanos. No se almacena ningún dato.
\end{UCtrayectoriaA}

\begin{UCtrayectoriaA}{B}{El Gerente de Recursos Humanos y/o Agente de Recursos Humanos no ingresa un dato que es obligatorio}
\UCpaso [\UCactor] Omite un dato obligatorio al dar de alta al nuevo empleado.
\UCpaso Se despliega la \IUref{IU12}{Error de inserción de datos}
\end{UCtrayectoriaA}

%--------------------------------------
% Puntos de extensión


%-------------------------------------- TERMINA descripción del caso de uso.